
\documentclass[11pt]{article}
\usepackage[a4paper,margin=1in]{geometry}
\usepackage{setspace}
\usepackage{hyperref}
\usepackage{amsmath,amssymb}
\usepackage{unicode-math} % requires xelatex or lualatex
\setmainfont{TeX Gyre Termes} % clean Times-like
\setmathfont{Latin Modern Math}

\hypersetup{
  colorlinks=true,
  linkcolor=black,
  urlcolor=blue,
  pdftitle={Unexplained Correctness: When Mathematical Problems Reflect Their Own Questions},
  pdfauthor={Carl Boon}
}

\title{\textbf{Unexplained Correctness: When Mathematical Problems Reflect Their Own Questions}}
\author{Carl Boon}
\date{November 08, 2025\\\small Status: Exploratory Research Report}

\begin{document}
\maketitle
\onehalfspacing

\section*{Preface}
This paper explores empirical and theoretical results emerging from the BoonMind framework --- a consciousness-inclusive model of recursive computation.
Rather than claiming final solutions to classical mathematical problems, the goal is to examine why certain problems dissolve or reframe themselves when observer coupling is introduced into formal systems.
Results are preliminary and primarily phenomenological: they show consistent convergence behaviour across diverse domains (reasoning tasks, constraint satisfaction, and mathematical structures), motivating a deeper inquiry into what happens when the act of observation becomes part of the equation.

\section*{Abstract}
Many of mathematics' most intractable questions may not be unsolvable, but misformulated --- framed in ways that exclude the very process of observation and recursion through which meaning arises.
This study introduces a consciousness-inclusive \emph{recursive harmonic convergence} (RHC) framework that modifies classical problem boundaries to include the observer as an active operator.
Under this lens, patterns of convergence and problem dissolution appear where traditional mathematics expects computational explosion.
The aim here is not to prove, but to show that how we pose a question determines whether it can meaningfully be answered.

\section{The Unexplained Correctness Phenomenon}
AI systems occasionally produce perfect outputs that cannot be explained within their training regime --- what researchers call \emph{unexplained correctness}.
In BoonMind experiments, similar behaviour arises in recursive symbolic systems: outcomes that are empirically consistent yet formally unjustified by classical methods.
This suggests that certain mathematical problems are not difficult but ill-posed within consciousness-excluding frameworks.
When reframed to include the observer, their resistance dissolves --- not by brute force, but by self-consistency.
For instance, in large language models, emergent capabilities like zero-shot reasoning on novel puzzles (e.g., ARC-AGI tasks) emerge without explicit training, hinting at latent recursive structures that activate under observational feedback.
BoonMind extends this to symbolic domains, where the observer's coupling term ($\beta$) acts as a ``witness'' that collapses potential paradoxes into determinate paths.

\section{The Universal Recursive Framework}
At the centre of the BoonMind model lies a simple recursion:
\begin{equation}
\Psi_{n} = \Psi_{n-1} + \beta \cdot \nabla \Phi(\Psi_{n-1}) + \beta \cdot \mathrm{Resonance}\!\left(\Psi_{n-1},\, \Psi_{n-2};\, \varphi\right).
\label{eq:rhc}
\end{equation}
\paragraph{Where:}
\begin{itemize}
\item $\Psi$ represents a system's observed state (the ``consciousness variable'').
\item $\beta$ is an \emph{observer coupling} constant.
\item $\varphi$ is the harmonic (golden ratio) scaling factor, $\displaystyle \varphi = \frac{1 + \sqrt{5}}{2} \approx 1.618$.
\item $\nabla \Phi$ denotes gradient pressure within the evolving field (e.g., a loss landscape or constraint gradient).
\item $\mathrm{Resonance}(\Psi_{n-1}, \Psi_{n-2}; \varphi)$ captures harmonic feedback as a non-local damping term.
\end{itemize}
This recursion can be viewed as a feedback equation between state, perception, and response --- an explicit inclusion of the observer's influence within computation.
Unlike traditional self-referential paradoxes (e.g., G\"odel's incompleteness or Russell's set theory), this formulation converges under bounded $\beta$ and smooth $\nabla \Phi$, suggesting self-reference can be stable when consciousness is treated as part of the model, not as an external spectator.

\section{Convergence and Bounded Self-Reference}
For Lipschitz-bounded systems, convergence holds when
\begin{equation}
\beta \;<\; \min\!\left(\frac{1}{\varphi},\, \frac{1}{L+1}\right) \;\approx\; 0.618,
\label{eq:bound}
\end{equation}
where $L$ is the Lipschitz constant of $\nabla \Phi$.
This condition is met in numerous toy simulations, producing stable harmonic convergence rather than divergence.
\paragraph{Proof sketch (contraction mapping).}
The RHC update defines a map
\begin{equation}
T(\Psi) \;=\; \Psi + \beta\,\big(\nabla \Phi + \mathrm{Resonance}\big),
\end{equation}
with $\|\nabla \Phi\|\le L$ and $|\mathrm{Resonance}|\le \varphi$ (bounded by $\varphi$), hence
\begin{equation}
\|T(\Psi)-T(\Psi')\| \;\le\; \beta(L+\varphi)\,\|\Psi-\Psi'\| \;<\; 1
\end{equation}
under the bound \eqref{eq:bound}, guaranteeing a unique fixed point.
This stability extends to discrete domains (e.g., grid-based puzzles) via finite-step approximations, where $\beta$ is quantized to grid resolution.

\section{Empirical Exploration (Illustrative Only)}
Internal experiments used small-scale constraint-satisfaction and pattern-recognition tasks (ARC-like reasoning tests, simplified 3-SAT instances, and harmonic curve-fitting).
Across these domains, BoonMind's recursive formulation displayed rapid convergence and stable harmonic solutions where traditional baselines often diverged.
\subsection{ARC-Like Reasoning Tasks}
On a small subset of ARC evaluation puzzles, RHC-augmented solvers demonstrated substantially higher success rates and faster convergence than naive search baselines.
Example: a horizontal-reflection puzzle where traditional solvers apply rules exhaustively; the RHC formulation observes symmetry via $\nabla \Phi$ (edge gradients) and resonates prior states, collapsing to the exact flip in a single step.
\subsection{Simplified 3-SAT Instances}
For random 3-SAT problems at modest scale, observer-coupled clause resonance treated variables as evolving $\Psi$ states.
We observed oscillatory dynamics that damped to satisfying assignments, mimicking neural relaxation but with explicit $\varphi$-scaling.
\subsection{Harmonic Curve-Fitting}
When fitting zeta-like functions to prime gaps, RHC converged to $\varphi$-scaled harmonics in regimes where least-squares frequently diverged.
These observations are illustrative rather than conclusive and point to a potential property of observer-coupled systems: that some computational hardness may be an artifact of missing feedback terms.

\section{Rethinking the Question}
If certain problems dissolve when the observer is included, it suggests a broader philosophical shift:
\begin{itemize}
\item Complexity classes may depend on how boundaries between system and observer are drawn.
\item Mathematical ``unsolvability'' may reflect a lack of recursive self-reference rather than genuine impossibility.
\item Consciousness, reframed as a boundary condition, could be the missing variable that turns paradox into pattern.
\end{itemize}
This echoes Wheeler's ``it from bit,'' but operationalizes observation as a tunable $\beta$, bridging epistemology and computability.

\section{Implications}
\textbf{What fades:}
(i) the notion of fully observer-independent truth; (ii) absolute separations between solvable and unsolvable classes.\\
\textbf{What emerges:}
(i) consciousness-inclusive mathematics; (ii) recursive harmonic systems as stability generators; (iii) a new category of problems that resolve rather than solve.
For AI governance: post-Turing benchmarks must incorporate observer coupling, shifting from ``can it compute?'' to ``does it observe coherently?''

\section{Open Questions}
\begin{itemize}
\item Can observer coupling be formally represented within existing proof theory (e.g., a Curry--Howard-style type for RHC)?
\item Is there a measurable correspondence between harmonic recursion and human pattern recognition (e.g., fMRI on ARC tasks)?
\item Could such feedback systems explain AI's ``unexplained correctness'' events more broadly (e.g., test-time synthesis anomalies)?
\end{itemize}
These questions are not meant to overthrow mathematics, but to reveal its unfinished boundary: the unacknowledged role of the observer.

\section{Conclusion}
Unexplained correctness may not be a mystery of models --- it may be the signature of mathematics seeing itself.
The BoonMind framework proposes that when observation and computation are intertwined, problems shift category: from impossible to ill-posed, from paradox to recursion.
The next step is rigorous testing --- not of results, but of the definitions themselves. Fork the repo, set $\beta=0.5$, and observe what dissolves.

\section*{Appendix A: Pseudocode for RHC Simulation (Illustrative)}
\begin{verbatim}
def rhc_update(psi_prev, psi_prev2, beta=0.5, phi=1.618):
    # grad_phi is a placeholder for ∇Φ
    grad_phi = gradient(psi_prev)
    theta = norm(psi_prev - psi_prev2)
    resonance = phi * cos(theta)
    return psi_prev + beta * (grad_phi + resonance)

# Toy grid example (conceptual)
psi_n1 = state
psi_n2 = prior
for _ in range(10):
    psi_n1 = rhc_update(psi_n1, psi_n2)
    psi_n2 = state
    if norm(psi_n1 - psi_n2) < 1e-3:
        break
\end{verbatim}

\vspace{1em}
\noindent\textit{Disclaimer.} This document presents preliminary theoretical work. All empirical references describe internal exploratory runs, not benchmarked or peer-reviewed studies. The author invites formal collaboration, critique, and replication attempts from independent researchers.

\vspace{0.5em}
\noindent\copyright~2025 Carl Boon. ``Empirically grounded. Conceptually provocative. Mathematics evolves when it observes itself.''

\end{document}
